The primary data set that we use is the 1990 Journey to Work (JTW) data, which is derived from the 1990 Census long form. In the JTW questions, employed respondents are asked ``at what location did this person work LAST WEEK (If this person worked at more than one location, print where he or she worked most last week.'' The 1990 JTW data geocodes these responses and publishes estimated home-to-work flows for all counties and county equivalents in the United States.\footnote{This data, as well as the ACS data described later, is available for download from the Census website, at \url{http://www.census.gov/hhes/commuting/data/commutingflows.html}} The Office of Management and Budget (OMB) uses JTW in its guidelines for defining Metropolitan Statistical Areas, a geographic unit used for research, statistical publications, and federal programs.\footnote{See the Federal Register Part IX, Office of Management and Budget: Standards for Defining Metropolitan and Micropolitan Statistical Areas; Notice, December 27, 2000, available  at \url{http://www.census.gov/population/metro/files/00-32997.pdf}} \citet{TS1996} use the 1990 JTW data for their analysis. We use the 1990 JTW data to replicate TS1996's Commuting Zone definitions, as well as estimate our alternate labor market definitions. 

In addition to the 1990 JTW data, we also use the 2000 JTW data, and the 2009-2013 5-Year Pooled American Community Survey JTW data. The ACS data are similar to the Decennial Census data, but include fewer pairs of counties in the published flows data. The JTW from ACS also includes published margins of error for these flows, whereas the 1990  and 2000 JTW data do not include these measures.

Table \ref{tab:comstat} provides summary statistics on the worker flows data from the three main data sources. We summarize the number of residence counties in the file, the total count of jobs included in the flows, the mean residence labor force and home-to-work flow sizes, and the mean share of workers who work in the same county where  they live. We note the rise in population, or total job count, from 1990 through 2009, as well as the declining share of within-county commuting during that period (a lower within-county share). The decline in the share of within-county commuting, which shows an increased distance of jobs from where workers live, is consistent both with studies of Census and ACS data showing increasing travel time in recent years \citep{ACS2011}. 

We use a number of other data sources to measure local labor market integration in Section \ref{sec:objfn}: county-level unemployment rates from the Bureau of Labor Statistics Local Area Unemployment Statistics series; and annual earnings data from the Bureau of Economic Affairs Regional Economic Indicators Series (REIS), both from 1990-1995, which aligns most closely with the timing of our commuting data.

% table cluster changes
\begin{table}\centering
\caption{Commuting Summary Statistics \label{tab:comstat}}
\begin{tabular}{lccccc}
\hline\hline
 & & Total & Mean & Mean & Mean \\ 
 & & Residential & Residential & Flows & Within-county \\ 
Dataset & Counties & Labor Force & Labor Force & Home-to-work & Share of Flows \\ 
\hline
\multicolumn{6}{l}{\emph{Journey-to-Work (JTW) Estimates}}\\
1990 Census & 3,105 &   115,004,732 & 460,966 & 392,408 & 0.76 \\ 
2000 Census & 3,107 &   128,170,381 & 444,149 & 369,102 & 0.74 \\ 
2009-2013 ACS & 3,143 & 139,679,674 & 505,765 & 419,784 & 0.73 \\ 
%\hline
%\multicolumn{6}{l}{\emph{LODES Data}}\\
%2006 & 3,125 &        124,088,835 & 460,908 & 335,992 & 0.60 \\ 
%2007 & 3,125 &        125,922,964 & 459,064 & 329,053 & 0.58 \\ 
%2008 & 3,125 &        126,100,097 & 463,321 & 330,606 & 0.57 \\ 
%2009 & 3,121 &        121,594,710 & 441,584 & 312,014 & 0.56 \\ 
%2010 & 3,126 &        123,193,014 & 445,503 & 313,439 & 0.56 \\ 
%2011 & 3,126 &        125,903,901 & 452,765 & 318,870 & 0.55 \\ 
%2012 & 3,022 &        125,554,526 & 456,308 & 321,988 & 0.55 \\ 

\hline
\multicolumn{6}{p{6in}}{\footnotesize \textit{Notes}: Authors' calculations, using public-use county-to-county flows from Journey to Work estimates for the 1990 Census, 2000 Census, and the 2009-2013 5-Year ACS.}\\
\end{tabular}
\end{table}

