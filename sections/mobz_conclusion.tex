Recent influential papers in labor economics have used commuting zones as an alternative definition to local labor markets. However, no one has carefully analyzed the methodology used to construct commuting zones and how it may impact empirical findings. Our paper contributes to this literature by analyzing this methodology and its implications for empirical applications.

We document that the commuting zone methodology is sensitive to uncertainty in the input data and parameter choices and we demonstrate how these features affect the resulting labor market definitions. Furthermore, we demonstrate that uncertainty in local labor market definitions also affects empirical estimates that use commuting zones as a unit of analysis. In future work, we want to explore other clustering methods, which are less history-dependent, as they may come to more globally optimal solutions. We also want to develop a measure with which to compare candidate zones against one another, using multiple metrics of local labor market integration.

