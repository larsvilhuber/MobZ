Recent influential papers in labor economics have used commuting zones as an alternative definition to local labor markets. However, no one has carefully analyzed how elements of how these commuting zones are constructed may affect empirical findings. Our paper contributes to this literature by analyzing this methodology and its implications. 

We document that the commuting zone methodology is sensitive to uncertainty in the input data and parameter choices and we demonstrate how these features affect the resulting labor market definitions. Furthermore, we demonstrate that uncertainty in local llabor market definitions also affects empirical estimates that use commuting zones as a unit of analysis. Finally, we develop a metric to compare competing local labor market definitions, as well as a method for obtaining alternative local labor market definitions. We show that our alternative definition out-performs commuting zones, as well as other local labor market definitions commonly used.

There are a number of questions and issues that we have not yet addressed in this research. First, we want to demonstrate that our approach is not as sensitive to the issues outlined for heirarchical clustering. Second, we want to compare how robust various local labor market definitions are over time. Finally, we want to allow our clustering procedure to use more information than commuting to form clusters. 
